\documentclass{report}
\title{Gochem User's guide}
\author{Raul Mera-Adasme}
\date{February, 18, 2006}
\makeindex
\begin{document}
\maketitle
\chapter{Reference}
\section{Introduction}
This guide assumes you can program in the Go language. It's easy!
Take the tour of go \it{http://tour.golang.org/}, and with just a bit more read
you will be ready to start.


A reference for all the functions in gochem can be found at
\it{http://godoc.org/github.com/rmera/gochem}

\subsection{What is gochem?}
Gochem is a library for chemistry, especially computational 
biochemistry, in go. It provides structures for handling atoms and 
molecules, reading/writing some common formats (currently xyz, PDB, 
xtc and dcd) and a few functions for geometrical manipulation of 
molecules.


\subsection{Design goals}


\begin{itemize} 
\item Simplicity (of the code).
\item Readability.
\item Assume that users can code (dont overcomplicate the code with 
	zillion simplifying functions. When some of these are provided 
	they should be separated from the rest.) The helper functions 
	provided are to be clearly identified and isolated from the core
	of the library.
\item Try to avoid the need for users to know gomatrix as much as possible
	-Supply convenience functions to deal with coordinates.
	-Of course users wanting to code calculations will need gomatrix 
	anyway.
\item Fast and light
\item Concurrent when possible and necessary
\item Easy to extend
\item Useful for computational biochemistry at a classical and QM levels.

\end{itemize}

\section{Using the library}

The library's core are a few go \it{interfaces} 

\begin{itemize}
\item The \bf{Ref} interface. It contains the information about the atoms in a systems, except for coordinates.
\item The \bf{Traj} interface. It contains the coordinates for all the states of the atoms of a system. The coordinates for each state are implemented as a go.matrix matrix.DenseMatrix (\it{https://github.com/skelterjohn/go.matrix}, reference at \it{http://godoc.org/github.com/skelterjohn/go.matrix})
\end{itemize}

The most important concrete implementation of Ref and Traj is the type \bf{Molecule}. Molecule is the type you obtain when you read a PDB or an XYZ file into gochem. Appart from implementing the Ref and Traj interfaces, it contains information about the b-factors of a molecule (information present in a PDB file).

Most gochem functions require a Ref interface and a Traj interface as arguments. Several others take a Traj and a DenseMatrix (i.e. the coordinates for one state). 

\subsection{An example workflow}

\begin{enumerate}

\item Load a system from a PDB or XYZ file, obtaining a chem.Molecule.
\item Collect a subset of coordinates which are of interest. (for instance, the backbone atoms of a protein)
\item Do something with the coordinates (study their distribution, or modify them as in rotating them about an axis)
\item If the coordinates were modified, resplace coordinates for the atoms of interest in the original molecule by the modified ones.
\item Print the results and, if applicable, generate a PDB or XYZ file with the new coordinates.

\end{enumerate} 





\end{document}
